\documentclass[final]{beamer}

% <<< packages

\usetheme[size=a1,orientation=landscape]{uvtposter}

\usepackage{amsmath}

\usepackage{hyperref}
\usepackage{booktabs}

\usepackage{listings}

% >>>

% <<< commands

\NewDocumentCommand \dx { O{x} } {\,\mathrm{d} #1}

% >>>

% <<< metadata

\title{West University of Timișoara: \\ A simple and modern Conference Poster Template}
\author{Hubert J. Farnsworth\inst{1} \and Amy Wong-Kroker\inst{2}}
\institute[shortinst]{\inst{1} West University of Timișoara \quad \inst{2} Mars University}

\footercontent{
    \href{https://info.uvt.ro/}{https://info.uvt.ro/} \hfill
    27th SYNASC 2025, Timișoara --- September 22--25 \hfill
    \href{mailto:info@e-uvt.ro}{info@e-uvt.ro}}

% >>>

\begin{document}

\begin{frame}[fragile]
\begin{columns}[t]

\separatorcolumn
\begin{column}{0.3\paperwidth}

\begin{block}{Introduction}
    The \textbf{UVT Conference Poster Theme} is a modern theme based on
    \texttt{beamerposter} that uses the official
    \textcolor{UVTLightBlue}{UVT branding}\footnotemark[1]. Some
    of its elements and styling are inspired by \textcolor{UVTLightBlue}{Gemini}
    theme\footnotemark[2].

    \bigskip
    It comes with a few nifty features:
    \begin{itemize}
        \item Use of the official color scheme for UVT throughout.
        \item Customized block environments.
        \item Easy to extend and modify separate parts (header, fonts, etc).
    \end{itemize}

    \bigskip
    Separate parts are available as e.g. \texttt{\textbackslash usecolortheme\{uvtposter\}}!
    \bigskip

    To use it in you documents add something like the following

    \begin{center}
    \begin{lstlisting}[language=TeX]
    \documentclass[final]{beamer}
    % 1. The theme supports different sizes, e.g. a0, a1, ...
    % 2. It can also be viewed in "landscape" or "portrait" mode
    \usetheme[size=a1,orientation=landscape]{uvtposter}

    % 3. You can select other color schemes or box types
    \usecolortheme{uvtposter}
    \end{lstlisting}
    \end{center}

    \bigskip
    More information is available in \texttt{template.tex}.
    \bigskip

    \footnotetext[1]{\url{https://dci.uvt.ro/identitate-vizuala}}
    \footnotetext[2]{\url{https://github.com/anishathalye/gemini/}}
\end{block}

\begin{block}{Colors and Fonts}
    This theme uses the \emph{Myriad Pro} font. This is a clean \emph{sans serif}
    font from Adobe that is recommended by the official UVT branding.

    \bigskip
    The theme has three standard colors:
    \begin{itemize}
        \item A nice \textcolor{UVTLightBlue}{\textbf{light blue}} (\texttt{UVTLightBlue}).
        \item A bolder \textcolor{UVTDarkBlue}{\textbf{dark blue}} (\texttt{UVTDarkBlue}).
        \item An attention grabbing \textcolor{UVTYellow}{\textbf{yellow}}
        (\texttt{UVTYellow})!
        \item Variants \texttt{color!x} can also be used to darken or lighten them
    \end{itemize}

    \bigskip
    These can be used for emphasizing \textcolor{UVTLightBlue}{text} or for
    more obvious \alert{alerts}. Standard \textbf{bold} and \textit{italic} emphasis
    can of course also be used!
\end{block}

\begin{alertblock}{Lists, enumerations, and descriptions}
\heading{Lists}

\begin{itemize}
    \item Itemize lists are nicely customized ...
    \begin{itemize}
        \item All the way down ...
        \begin{itemize}
            \item To the third level!
        \end{itemize}
    \end{itemize}
\end{itemize}

\heading{Enumerate}

\begin{enumerate}
    \item We can also enumerate!
    \item Many things!
    \item It's great!
\end{enumerate}

\heading{Description}

\begin{description}
    \item[UVT] And describe our university in exquisite detail, so that we
        can capture all its multiline greatness!
\end{description}
\end{alertblock}
\end{column}

\separatorcolumn

\begin{column}{0.3\paperwidth}
\begin{block}{Blocks}

    We have various standard Beamer blocks styled in a pleasing fashion. You
    might have noticed their use for each of the "sections" up to now.

    \bigskip

    \begin{block}{Block 1}
        Some block block: \verb|\begin{block} ... \end{block}|.
    \end{block}

    \begin{alertblock}{Alert 1}
        Some alert block: \verb|\begin{alertblock} ... \end{alertblock}|.
    \end{alertblock}

    \begin{exampleblock}{Example 1}
        Some example block: \verb|\begin{exampleblock} ... \end{exampleblock}|.
    \end{exampleblock}

    \bigskip

    You probably should not nest these as we just did here!
\end{block}

\begin{block}{Figures}
    You can also add figures to each block.
    \bigskip

    \begin{figure}
    \begin{tikzpicture}[scale=2.8]
        % Draw axes
        \draw [<->,thick] (0,2) node (yaxis) [above] {$y$}
            |- (3,0) node (xaxis) [right] {$x$};
        % Draw two intersecting lines
        \draw (0,0) coordinate (a_1) -- (2,1.8) coordinate (a_2);
        \draw (0,1.5) coordinate (b_1) -- (2.5,0) coordinate (b_2);
        % Calculate the intersection of the lines a_1 -- a_2 and b_1 -- b_2
        % and store the coordinate in c.
        \coordinate (c) at (intersection of a_1--a_2 and b_1--b_2);
        % Draw lines indicating intersection with y and x axis. Here we use
        % the perpendicular coordinate system
        \draw[dashed] (yaxis |- c) node[left] {$y'$}
            -| (xaxis -| c) node[below] {$x'$};
        % Draw a dot to indicate intersection point
        \fill[UVTYellow] (c) circle (2pt);
    \end{tikzpicture}
    \caption{A simple figure.}
    \end{figure}

    \bigskip

    These should be used quite extensively to better make your point. This is
    a visual sort of medium after all!
\end{block}

\begin{block}{Tables}
    \begin{table}
    \centering
    \caption{Largest cities in the world (source: UN 2018 population estimates
    from Wikipedia)}
    \begin{tabular}{@{} lr @{}}
        \toprule
        City & Population\\
        \midrule
        Tokyo & 37,468,000 \\
        Delhi & 28,514,000 \\
        Shanghai & 25,582,000 \\
        São Paolo & 21,650,000 \\
        \bottomrule
    \end{tabular}
    \end{table}

    \bigskip

    Tables are also a good way to display data. Using the \texttt{booktabs}
    package should make your tables look very nice indeed!
\end{block}

\end{column}

\separatorcolumn

\begin{column}{0.3\paperwidth}

\begin{exampleblock}{Example: Infinite Primes}
    Imagine that there were only a limited number of prime numbers, and we could
    list them as \(p_1, p_2, \ldots, p_n\). Now, consider the number formed by
    multiplying all these primes together and then adding one, that is,
    \[
    N = p_1 \times p_2 \times \cdots \times p_n + 1.
    \]

    \bigskip

    If you try to divide this new number \(N\) by any of the primes on your list,
    you always get a remainder of 1. This means that none of the primes
    \(p_1, p_2, \ldots, p_n\) can be a divisor of \(N\). Since every number
    greater than 1 must have a prime factor, \(N\) must either be a prime
    itself or be divisible by a prime that was not in our original list.

    \bigskip

    This simple observation shows that no matter how many primes you start with,
    you can always construct a number that reveals at least one additional prime,
    proving that there are infinitely many prime numbers.
\end{exampleblock}

\begin{exampleblock}{Example: Grönwall Inequality}
    Suppose you have a nonnegative function \(u(t)\) that satisfies an inequality
    of the form
    \[
        u(t) \le a + \int_{t_0}^t b(s)\, u(s) \dx[s],
    \]
    where \(a \ge 0\) is a constant and \(b(t) \ge 0\) is a given function. The
    idea behind the inequality is that even though \(u(t)\) might depend on its
    past values through the integral, its growth is controlled by the accumulation
    of the function \(b(t)\).

    \bigskip
    To see how this control works, define an auxiliary function
    \[
        v(t) = a + \int_{t_0}^t b(s)\, u(s) \dx[s].
    \]

    Since \(u(t) \le v(t)\), differentiate \(v(t)\) with respect to \(t\) to obtain
    \[
        v'(t) = b(t)\, u(t) \le b(t)\, v(t).
    \]

    Dividing both sides by \(v(t)\) (which is positive) gives
    \[
        \frac{v'(t)}{v(t)} \le b(t).
    \]

    Integrating this inequality from \(t_0\) to \(t\) and exponentiating yields
    \[
        v(t) \le v(t_0) \exp\!\left(\int_{t_0}^t b(s)\, \dx[s]\right).
    \]

    Since \(v(t_0)=a\) and \(u(t) \le v(t)\), we conclude that
    \[
        u(t) \le a\, \exp\!\left(\int_{t_0}^t b(s)\, \dx[s]\right).
    \]

    \bigskip

    This simple argument shows that if the function \(u(t)\) satisfies the original
    integral inequality, then its value at any time \(t\) is bounded above by an
    exponential function determined by the integral of \(b(t)\); this is precisely
    the statement of the integral Grönwall inequality.
\end{exampleblock}

\begin{alertblock}{References}
    \nocite{*}
    \bibliographystyle{plain}
    \bibliography{template}
\end{alertblock}

\end{column}
\separatorcolumn
\end{columns}
\end{frame}

\end{document}
